%%%%%%%%%%%%%%%%%%%%%%%%%%%%%%%%%%%%%%%%%
% Masters Degree Thesis 
% Use pdflatex to compile  
% Note:
% All document variables are stored in the Thesis.cls file
%
%%%%%%%%%%%%%%%%%%%%%%%%%%%%%%%%%%%%%%%%%

%----------------------------------------------------------------------------------------
%	PACKAGES AND OTHER DOCUMENT CONFIGURATIONS
%----------------------------------------------------------------------------------------

\documentclass[11pt, oneside]{Thesis} % The default font size and one-sided printing

\graphicspath{{Pictures/}} % The directory where pictures are stored

\usepackage[square, numbers, url, comma, sort&compress, tocloft, cite]{natbib}
\usepackage[a4paper,lmargin=2.6in,rmargin=0in,tmargin=2.5in,bmargin=0in]{geometry}
\usepackage{sectsty}
\usepackage[center]{titlesec}
\hypersetup{urlcolor=black, colorlinks=true} % Colors hyperlinks in black
\title{\ttitle} % Defines the thesis title
\newcommand*{\justifyheading}{\raggedright}
\newcommand*{\justifyhead}{\centering}
\titleformat{\chapter}[display]
  {\normalfont\huge\bfseries\justifyhead}{\chaptertitlename\ \thechapter}
  {20pt}{\Huge}
\titleformat{\section}
  {\normalfont\Large\bfseries\justifyheading}{\thesection}{1em}{}
\titleformat{\subsection}
  {\normalfont\large\bfseries\justifyheading}{\thesubsection}{1em}{}
\begin{document}

%----------------------------------------------------------------------------------------
%	TITLE PAGE
%----------------------------------------------------------------------------------------
\frontmatter % Use Roman page numbering style (i, ii, iii, iv...) for pre-content pages

\begin{titlepage}
\begin{center}
\textsc{\Large \textbf{UNIVERSITY OF BUEA}}\\[25pt]\\[3.0cm] % University of Buea
\end{center}

\begin{minipage}{0.5\textwidth}
\begin{flushleft}
\large\textbf{FACULTY OF SCIENCE}
\end{flushleft}
\end{minipage}
\begin{minipage}{0.5\textwidth}
\begin{flushright}
\large\textbf{DEPARTMENT OF COMPUTER}\\  
\end{flushright}
\end{minipage}
\large \hspace*{275} \textbf{SCIENCE}
\hspace{200}\\

\begin{center}
\\[1.5cm]
{\large \bfseries IMPLEMENTATION OF A HEART-SHAPED PRIMITIVE}\\[0.4cm] % Implementation of a Heart-Shaped Primitive
\\[2.5cm]

\large \textbf{ By } \\[0.75cm]

\Large \textbf{\authornames} \\ 

\small B.Sc. (Hons) Mathematics \\[2.5cm]

\large \textbf{A Thesis Submitted to the Department of Computer Science, \\
		Faculty of Science, University of Buea, in Partial Fulfilment \\ 
		of the Requirements for the Award of the \\
		Master of Science (M.Sc.) Degree \\
		in Computer Science}\\[2.5cm]

\large \textbf{July, 2015}

\end{center}

\end{titlepage}

%----------------------------------------------------------------------------------------
%	DEDICATION PAGE
%----------------------------------------------------------------------------------------
\Dedication{
\pagestyle{plain} % No headers or footers for the following pages
\begin{center}\Large To My Family \end{center}

\vfill\vfill\vfill\vfill\vfill\vfill\null % Add some space at the bottom to position the quote just right
}
\clearpage % Start a new page

%----------------------------------------------------------------------------------------
%	CERTIFICATION PAGE
%----------------------------------------------------------------------------------------

\begin{center}
\textsc{\Large \textbf{UNIVERSITY OF BUEA}}\\[25pt]\\[0.5cm] % University of Buea
\end{center}

\begin{minipage}{0.5\textwidth}
\begin{flushleft}
\large\textbf{FACULTY OF SCIENCE}
\end{flushleft}
\end{minipage}
\begin{minipage}{0.5\textwidth}
\begin{flushright}
\large\textbf{DEPARTMENT OF COMPUTER}
\end{flushright}
\end{minipage}
\large \hspace*{275} \textbf{SCIENCE}

\Certification{

\addtocontents{toc} % Add a gap in the Contents

The thesis of \authornames\hspace{2}(SC09B676) entitled: “\textbf{\ttitle}” submitted to the \deptname, \facname\hspace{4}at the \univname\hspace{3} in partial fulfillment of the requirements for the award of the Master of Science (M.Sc.) degree in Computer Science has been examined and approved by the examination panel composed of:\\

\hspace*{1em}Ngwa A. Gideon (Ph.D), Chairperson ( Associate Professor of Mathematics )\\
\hspace*{1em}Nkweteyim L. Denis (Ph.D), Examiner ( Lecturer of Computer Science )\\
\hspace*{1em}Kamgnia Emmanuel (Ph.D), Supervisor (Lecturer of Computer Science )\\[0.5cm]

\begin{minipage}{0.5\textwidth}
\begin{flushleft}
\Large \rule[1em]{11em}{1.0pt}
\end{flushleft}
\end{minipage}
\begin{minipage}{0.5\textwidth}
\begin{flushright}
\Large \rule[1em]{11em}{1.0pt} 
\end{flushright}
\end{minipage}
\Large \hspace{95} Nkweteyim L. Denis (PhD) \hspace{80} Kamgnia Emmanuel (PhD)\\
\large\hspace*{2em} (Head of Department) \hspace{155} (Supervisor)\\[1.0cm]

\Large This thesis has been accepted by the Faculty of Science.\\[0.5cm]

\Large \textbf{Date:}\\
\begin{minipage}{0.5\textwidth}
\begin{flushleft}
\Large \hspace*{3em}\rule[1em]{10em}{1.0pt}
\end{flushleft}
\end{minipage}
\begin{minipage}{0.5\textwidth}
\begin{flushright}
\Large \rule[1em]{10em}{1.0pt} 
\end{flushright}
\end{minipage}
\hspace*{19em}Prof. Ayonghe N. Samuel\\ 
\hspace*{23em}( Dean )
}
\clearpage % Start a new page

%----------------------------------------------------------------------------------------
%	ACKNOWLEDGEMENTS
%----------------------------------------------------------------------------------------
\Acknowledgement{

\addtocontents{toc}% Add a gap in the Contents,
\\
Firstly, I thank our Lord Jesus Christ for giving me
life and good health during my stay at University Of Buea and for
granting me the patience to go through this program.

I am grateful to the Vice Chancellor of the University Of Buea and her collaborators for 
maintaining an enabling environment where graduate students can  
learn and express their utmost research potential. I am especially indebted  
to Professor Joyce B.M. Endeley, former Director of the Higher  
Technical Teachers’ Training College of the University of Buea in Kumba  
and Professor Theresa Nkuo­ Akenji, the Deputy Vice Chancellor for  
Internal Control and Evaluation who authorized the connection of internet  
facilities to the Postgraduate Computer Science Laboratory during the  
implementation of this project.
I wish to sincerely thank all the staff of the Department of Computer  
Science for their advice and stimulating conversations during the writing of  
this thesis. Special thanks go to Dr. Denis Nkweteyim and Professor Emmanuel Kamgnia 
for meticulously proofreading through and correcting this work while 
making their wide spectrum of research experience and books  
on graphics available to me.

My appreciation goes to the Google Open Source Programs
Office (OSPO) – Mary Radomile, Carols Smith, Cat Allman and
Stephanie Taylor, as well as the United States Army Research Laboratory BRL-CAD team
 for their sponsorship of this project under the auspices
of the 2013 Google Summer Of Code program. Special thanks go
to Mr.Christopher Sean Morrison and Mr.Erik Greenwald for their
valuable advice and mentorship during the implementation of this
project. I am also indebted to Juoko Orava aka \textit{Nominal Animal} and the
mathematician Dr.Titus Piezas for tips on ray-tracing the heart
surface.

Many thanks go to my family for their assistance and
encouragement throughout this program.This thesis is dedicated to them.
I also thank the friends I made during the
course of this program whether from University of Buea or the Developer community,
for bringing all the fun they came along with– I appreciate your
company and I'm happy to know I was not alone.
}
\clearpage % Start a new page

%----------------------------------------------------------------------------------------
%	ABSTRACT PAGE
%----------------------------------------------------------------------------------------
\Abstract{

\abstract{\addtocontents{toc}%Add a gap in the Contents

This thesis titled \ttitle \hspace{2} was aimed at demonstrating the engineering of a heart-shaped primitive within the BRL-CAD ( Ballistic Research Laboratory Computer Aided Design ) package. Using a case study approach, the heart-shaped primitive's data structure was designed, necessary callback functions were written and tested with BRL-CAD's testing infrastructure. It was shown that the Laguerre-based root solver is indeed a sure-fire iterative method for finding roots of polynomials and it's stability on sextic equations was ascertained. Also, this work provided a guideline for the development of primitives within open source Computer Aided Design (CAD) software and showed that the BRL-CAD package primarily developed for the military sector can be used in non-military locales for entertainment purposes. \\[0.5cm]
\textbf{Keywords: } \keywordnames
}
}
\clearpage % Start a new page

%----------------------------------------------------------------------------------------
%	TABLE OF CONTENTS
%----------------------------------------------------------------------------------------

\pagestyle{plain} % Use the "plain" headers as defined before to bring them back
\lhead{\emph{Table of Contents}}

\Large \hspace*{125} \textbf{TABLE OF CONTENTS}\\
\large\\
Dedication ....................................................................................................... i\\
Certification .................................................................................................... ii\\
Acknowledgements ......................................................................................... iii\\
Abstract ......................................................................................................... iv\\
Table of Contents ........................................................................................... v\\
List of Tables ................................................................................................. vii\\
List of Figures ................................................................................................ viii\\
List of Abbreviations ..................................................................................... ix\\
\hspace*{155}			\textbf{CHAPTER ONE}\\ 
\hspace*{150}			\textbf{INTRODUCTION}\\
1.1 Historical Background ............................................................................... 1\\
1.2 Importance Of This Work ......................................................................... 3\\
1.3 Thesis Organisation ................................................................................... 4\\
\hspace*{155}			\textbf{CHAPTER TWO}\\
\hspace*{30}			\textbf{LITERATURE REVIEW OF GEOMETRIC MODELING}\\
2.1 Representation Schemes ........................................................................... 6\\
2.2 Wireframe Modeling ................................................................................. 8\\
2.3 Surface Modeling ...................................................................................... 9\\
2.3.1 Implicit Representation .......................................................................... 10\\
2.3.2 Parametric Representation ..................................................................... 10\\
2.3.3 Implicitization ........................................................................................ 11\\
2.3.4 Parameterization .................................................................................... 12\\
2.4 Solid Modeling .......................................................................................... 13\\
2.4.1 Boundary Representations (B-REPS) ..................................................... 13\\
2.4.2 Constructive Solid Geometry (CSG) ..................................................... 15\\
2.4.3 Spatial Subdivision ................................................................................ 16\\
2.5 Non-manifold Geometry ............................................................................ 21\\
\hspace*{150}			\textbf{CHAPTER THREE}\\
\hspace*{150}		\textbf{ANALYSIS & DESIGN}\\
3.1 The Open Source Community ................................................................... 23\\
3.2 Analysis Of The Work ............................................................................... 24\\
3.3 Design ....................................................................................................... 27\\
3.3.1 Tagging the heart-shaped primitive ......................................................... 27\\
3.3.2 Data Structure of Heart-shaped primitive .............................................. 28\\
3.3.3 A bare bones heart-shape ....................................................................... 29\\
3.3.4 Formatted description of the heart-shaped primitive .............................. 30\\
3.3.5 Database importation and exportation of the heart-shaped primitive ... 30\\
3.3.6 Computing the bounding box of the heart-shaped primitive ................. 31\\
3.3.7 Plotting the wireframe of the heart-shaped primitive ............................ 32\\
3.3.8 Surface representation and raytracing of the heart-shaped primitive .... 33\\
3.3.9 Type in support for the heart-shaped primitive ..................................... 36\\
\hspace*{150}			\textbf{CHAPTER FOUR}\\
\hspace*{130}		 \textbf{RESULTS & DISCUSSION}\\
4.1 Type-in support for the heart-shaped primitive ......................................... 37\\
4.2 Formatted description of the heart-shaped primitive ................................ 39\\
4.3 The bounding box of the heart-shaped primitive ......................................  40\\
4.4 Plotting the wireframe of the heart-shaped primitive ............................... 42\\
4.5 Ray tracing and surface representation of the heart-shaped primitive ...... 42\\
\hspace*{150}			\textbf{CHAPTER FIVE}\\
\hspace*{155}			\textbf{CONCLUSIONS}\\
5.1. Our Contribution ..................................................................................... 45\\
5.2. Further Research ...................................................................................... 45\\
\\
Bibliography ................................................................................................... 50\\
APPENDIX .................................................................................................... 50\\
%\tableofcontents % Write out the Table of Contents

\clearpage

%----------------------------------------------------------------------------------------
%	LIST OF TABLES
%----------------------------------------------------------------------------------------
\newgeometry{a4paper,lmargin=2.6in,rmargin=0in,tmargin=2.5in,bmargin=0in}
\Listoftables{
\pagestyle{plain} % No headers or footers for the following pages
Table 2.1: Implicit and parametric equations of some BRL-CAD primitives. . . . . 12 % Write out the List of Tables

\vfill\vfill\vfill\vfill\vfill\vfill\null % Add some space at the bottom to position the quote just right
}
\clearpage % Start a new page

%----------------------------------------------------------------------------------------
%	LIST OF FIGURES
%----------------------------------------------------------------------------------------

\Listoffigures{
\pagestyle{plain} % No headers or footers for the following pages
Figure 1.1: Model of a Goliath tracked mine . . . . . . . . . . . . . . . . . . . . . . . . 5\\
Figure 2.1: A wireframe of a sphere . . . . . . . . . . . . . . . . . . . . . . . . . . . . . 8\\
Figure 2.2: BRL-CAD Solid Primitive Shapes . . . . . . . . . . . . . . . . . . . . . . . .9\\
Figure 2.3: The Utah Tea Pot model . . . . . . . . . . . . . . . . . . . . . . . . . . . . .15\\
Figure 2.4: CSG Boolean operations . . . . . . . . . . . . . . . . . . . . . . . . . . . . .16\\
Figure 2.5: An M1A1 tank on a pedestal in a mirrored showcase room . . . . . . . . 17\\
Figure 2.6: Model of a Sphere Flake . . . . . . . . . . . . . . . . . . . . . . . . . . . . .20\\
Figure 2.7: Examples of Non-Manifold Geometry . . . . . . . . . . . . . . . . . . . . .21\\
Figure 3.1: Diagram of the heart-shaped primitive . . . . . . . . . . . . . . . . . . . . 28\\
Figure 4.1: Testing type in support for the heart-shaped primitive in archer . . . . 38\\
Figure 4.2: Testing the formatted description of the heart-shaped primitive. . . . . 39\\
Figure 4.3: Testing the bounding box of the heart-shape. . . . . . . . . . . . . . . . .41\\
Figure 4.4: Testing the wireframe of the heart-shaped primitive . . . . . . . . . . . . 42\\
Figure 4.5: Images rendered after raytracing the heart-shaped primitive . . . . . . . 44\\

\vfill\vfill\vfill\vfill\vfill\vfill\null % Add some space at the bottom to position the quote just right
}
\clearpage % Start a new page

%----------------------------------------------------------------------------------------
%	ABBREVIATIONS
%----------------------------------------------------------------------------------------
\Abbreviations{

\addtocontents{toc}%Add a gap in the Contents, for aesthetics

\setstretch{1.5} % Set the line spacing to 1.5, this makes the following tables easier to read
\Large \textbf{CAD :} \hspace{35} Computer Aided Design \\
\Large \textbf{BRL-CAD :} Ballistic Research Laboratory Computer Aided \\ \hspace*{90}Design\\
\Large \textbf{CSG :} \hspace{38} Constructive Solid Geometry \\
\Large \textbf{NURBS :} \hspace{13} Non-Uniform Rational B-Splines \\
\Large \textbf{NMG :} \hspace{30} Non - Manifold Geometry \\
\Large \textbf{B-REPS :} \hspace{12} Boundary Representations \\
}
\clearpage % Start a new page

%----------------------------------------------------------------------------------------
%	THESIS CONTENT - CHAPTERS
%----------------------------------------------------------------------------------------

\mainmatter % Begin numeric (1,2,3...) page numbering
% Include the chapters of the thesis as separate files from the Chapters folder

%-----------------------------------------------------------------------------------------------------------
\chapter{INTRODUCTION} % Main chapter title
\label{Introduction}
\lhead{Chapter 1. \emph{Introduction}} % The header on each page
%-----------------------------------------------------------------------------------------------------------

\section{Historical Background}
\hspace{30}Throughout our long history, we humans have always sought for means 
to express our creativity – ways to communicate our ideas through writing, 
sculpting, painting, carving, architecture and drawing. As a matter of fact,
paleolithic cave representations of animals at least 32,000 years ago in  
Southern France, ink drawings and paintings of human figures as well as writings 
in hieroglyphics on papyrus in the pyramids of ancient Egypt is
indicative of the fact that our need to express our individuality goes back to
antiquity. Before the renaissance, drawing was treated as a preparatory stage
for painting and sculpting. The wide availability of drawing instruments such as 
pens and pencils and most especially paper made master draftsmen like
Leonardo Da Vinci, Raphael and Michelangelo around the world to lift drawing
to an art in its own right. Thus, drawing stood out as the most popular and
fundamental means of public expression in human history and is one of the
simplest and most efficient means of communicating visual ideas. \cite{1}\\

\hspace{30}The drawing board era where paper, pens, pencils, rulers and ink  
prevailed has been relegated to the background in this information age which is  
powered by ubiquitous computer technology. Craving the unity of science and  
art, this essentially binary­-sequenced revolutionary device called the computer  
married the artistic and engineering forms of drawing into an androgynous one  
called Computer­ Aided Geometric Design or geometric modeling for short.
Geometric modeling currently involves the use of computers to aid in the  
creation, manipulation, maintenance and analysis of representations of the  
geometric shapes of two and three-­dimensional objects \cite{2}. It is the outgrowth  
of convergent motivations and developments from several works of life as  
outlined below.
\begin{itemize}
\item In the 1950s, the need to automate the engineering drawing process led
to electronic drawings which could be archived and modified more easily,
could be easily verified and errors could be eliminated from mechanical 
designs without introducing new ones. These computer drafting systems
allowed designers to produce drawings of objects by projecting
three-­dimensional objects unto two-­dimensional surfaces.
\item Then in the 1960s, there was a pressing need for software in the
automobile, shipbuilding and aircraft industries to produce
computer-­compatible descriptions of geometric shapes which can be
machined from wood and steel into stamps and dies for the
manufacturing and assembling of car parts,ship hulls as well as wings and  
fuselages using computer numerically controlled tools.
\item Later in the 1970s, the growing need for computers to render realistic
images of objects as well as animate solid objects pushed research
institutes like Xerox Palo Alto Research Center (Xerox PARC) and Apple
Computers to make significant contributions to graphical user­ interface
design and computer graphics.  
\end{itemize}

These needs and problems could only be solved by research in fields such as
graphics, animation and applications from algebraic geometry. The work of
various computer scientists and mathematicians lead to the active development
of several commercial packages sponsored by companies such as Renault,
Citroen, Ford and Boeing who could afford the computers capable of  
performing such lengthy calculations.  

\hspace{30}Today, geometric modeling is also referred to as Computer ­Aided Design
(CAD), is pronounced “kad” and is routinely used in the design and
manufacturing of engineering and architectural structures such as buildings, car  
parts, ship hulls and aircraft artillery as well as to specify special effects in  
cartoon movies, music videos and television shows. Indeed, CAD packages
provide facilities for designing shapes of solid physical objects and specifying
their motion in a way that art and science can unite to create cool designs.  

\hspace{30} Even though significant progress had been made in basic research and
the functionality of commercially available solid modelers like Apple
Computer's RenderMan, many solid modelers especially within the open
source community are still limited in their geometric features. The open
source community is a self organising collaborative social network of 
programmers driven by a passion to solve problems using computers.
It has several thousands of its projects on sites that offer services 
like bug tracking, mailing lists and version control viz \href{https://github.com/}{Github} 
and \href{http://sourceforge.net}{Source Forge}. These projects are constantly 
being improved upon by thousands of programmers putting in time and effort 
to write and debug software without direct monetary pay.

\hspace{30} In this thesis, we document the process of developing a heart­-shaped
primitive, a set of callback functions and procedures which compute geometrically 
useful properties of solids such as wireframe plotting, database importation and 
exportation, ray tracing, bounding box calculations, just to name a few, within the
 Ballistic Research Laboratory Computer Aided Design (BRL-­CAD) software package.  

\hspace{30} BRL-­CAD was initiated by the United States Army Research Laboratory
in 1983, the same agency which created the E.N.I.A.C., the world's first 
general ­purpose computer in the 1940s, to model military systems for the
United States Government. According to \cite{3}, BRL-­CAD became born again in 
2004 when it joined the open source community with portions of its source
code licensed under the Lesser General Public License (LGPL) and Berkeley
Software Distributions (BSD) licenses and has been credited as being the
oldest open source repository in continuous development. It supports a wide
variety of geometric representations including an extensive set of traditional
implicit primitive shapes as well as explicit primitives made from collections of
uniform B­spline surfaces, Non­uniform Rational B­spline (NURBS) surfaces,  
Non­manifold geometry (NMG) and purely faceted polygonal mesh geometry.  

\hspace{30} BRL­-CAD also focuses on solid modeling aspects of Computer ­Aided
Design. Figure 1.1 below shows a three-­dimensional model of a Goliath tracked
mine, a German engineered remote controlled vehicle used during World War II.
This model was created by students new to BRL-­CAD in the span of about 2
weeks, starting from actual measurements in a museum.

%----------------------------------------------------------------------------------------

\section{Importance Of This Work}

This work is significant to several stakeholders for several reasons ;

\begin{itemize}
\item By ray­tracing the heart's surface, it demonstrates to the scientific
community that the Laguerre zero­finder is indeed a sure­fire iterative
method for finding roots of polynomials and that Laguerre-­based root 
solvers are stable on sextic equations.
\item This work incorporates more geometric modeling functionality into 
the free and open source software community through BRL-­CAD, the oldest open 
source repository in continuous development \cite{3} by going beyond traditional 
CSG primitives shapes such as tori, spheres, boxes and ellipsoids towards the more
complex heart-shape based on a sextic equation. This work provides a guideline 
for the development of primitives within open source CAD software.
\item Given that BRL­-CAD is used within governments to model military artillery
 and for engineering and analysis purposes within academia, this
heart­-shaped primitive gives BRL-­CAD a loving aura – an 
environment where artists can produce cartoon animations as well as
design cards, royal seals and banners, gifts and presents for family and
communal celebrations such as weddings, family reunions and
Valentine's day for entertainment purposes.
\end{itemize}

%--------------------------------------------------------------------------------------------

\section{Thesis Organisation}

This thesis is divided into five (5) chapters. Chapter 1 introduces the study and chapter 2 reviews the literature in the field of geometric modeling. In Chapter 3, we state the problems we intend to solve and our project design. In chapter 4, we discuss the interesting results which we obtained. Finally, in
chapter 5, we state the contribution of our work and give possible research
directions which can proceed from it.

%--------------------------------------------------------------------------------------------

\begin{figure}[htbp]
\centering
\includegraphics[trim=1cm 2cm 3cm 4cm, clip=true, totalheight=0.5\textheight]{Figures/Goliath.png}
\caption[Model of a Goliath tracked mine]{Model of a Goliath tracked mine}
\label{Goliath}
\end{figure}

%--------------------------------------------------------------------------------------------

% Chapter 2

\chapter{LITERATURE REVIEW OF GEOMETRIC MODELING} % Main chapter title

\label{Literature Review} % Change X to a consecutive number; for referencing this chapter elsewhere, use \ref{ChapterX}

\lhead{Chapter 2. \emph{Literature Review}} % Change X to a consecutive number; this is for the header on each page - perhaps a shortened title

%----------------------------------------------------------------------------------------
%      INTRODUCTORY PARAGRAPHS
%----------------------------------------------------------------------------------------
\hspace{30} With the advent of computers which could perform millions of floating
point operations in unit time and which are still growing faster, researchers who
believed computers could aid the processes of mechanical design and
manufacturing were faced with a critical issue – how to represent physical
reality using computer software. They sought the best data structures to
represent this reality and the most appropriate algorithms to manipulate these representations.

\hspace{30} BRL-­CAD supports a wide variety of geometric representations including an 
extensive set of traditional implicit primitive shapes as well as explicit primitives
made from collections of uniform B­spline surfaces, non­uniform rational B­spline (NURBS)
surfaces, n­on-manifold   geometry   (NMG)   and purely faceted polygonal mesh geometry.
Consequently, in this chapter, we review the existing work done by scholars in the 
field of geometric modeling which have been applied to the development of BRL-­CAD. 
First of all, it introduces the issue of representation and the notion of representation
 schemes.Then, it summarizes developments in wireframe modeling, surface modeling, solid
modeling and non­-manifold modeling (aka non­manifold geometry or nmg for short) with a keen
 eye on the algorithms underlying them.

\hspace{30} As we progress in our literature review from older forms of geometric
modeling to newer ones, we will discover that representation schemes were
closely linked to algorithmic efficiency and that it has always been normal to
expect designers to switch to newer ones in response to the improvements in
algorithmic performance.Despite these enhancements in algorithmic efficiency
within the designer community, we cannot say with complete certainty whether
traditional representation schemes can be relegated to the background. We
can only conclude that old and new representation paradigms co­exist and that
research led to representation schemes which supplemented the repertoire of
geometric modeling.  

%-----------------------------------------------------------------------------------------

%----------------------------------------------------------------------------------------
%	SECTION 1
%----------------------------------------------------------------------------------------

\section{Representation Schemes}

A representation $\textbf{\mathfrak{R}}$ of a solid or representation for short is a subset of
three­-dimensional Euclidean space denoted $\mathbb{E}^3$ which models a physical solid.  
According to [5], Requicha and Tilove stated that point set topology provided a
formal language for describing the geometric properties of solids and they also  
threw more light on the mathematical characteristics of solids such as a solid's
interior, boundary, complement, closure, boundedness and regularity.
Requicha [4] insisted that to be computationally useful, a representation should  
formally capture the following properties ;

\begin{itemize}
\item \textit{\textbf{Rigidity:}} Representations should have an invariant configuration
irrespective of their location and orientation.
\item \textit{\textbf{Homogeneity:}} A representation should have an interior.
\item \textit{\textbf{Finiteness:}} A representation must occupy a finite amount of space.
\item \textit{\textbf{Boundary   determinism:}} A representation must unambiguously determine
the interior of that solid.
\item \textit{\textbf{Closure}}: Representations of solids which are manipulated by rigid
motions and regularized boolean operations should produce representations of solids too.
\end{itemize}
These formal characteristics leave representations no choice than to be
bounded, closed, regularized and semi­analytic, hence their coinage r­sets
according to [5]. An \textit{\textbf{r-­set}} is simply a regular and bounded set in $\mathbb{E}^3$.

\hspace{30} A representation scheme is simply a relation between physical solids and their representations which can be characterized by the following properties;
\begin{itemize}
\item \textit{\textbf{Domain}}: A representation scheme must represent quite a number of useful geometric solids. 
\item \textit{\textbf{Unambiguity}}: A representation scheme should produce representations which intuitively capture the properties of the physical solid so that it can be easily distinguished from other representations.
\item \textit{\textbf{Uniqueness}}: A representation scheme should uniquely represent a solid object within a software's database.  
\item \textit{\textbf{Validity}}: Representation schemes should yield representations of solids which do not exist or are valid.  
\item \textit{\textbf{Closure}}: A Representation scheme which transforms (reflects, scales, rotates) a representation should yield other representations too.
\item \textit{\textbf{Compactness}}: Representation schemes should yield representations which save space and allow efficient algorithms to   determine desirable physical characteristics.
\end{itemize}
%--------------------------------------------------------------------------------------------------
%	WIREFRAME MODELING
%-----------------------------------------------------------------------------------------------------
\section{Wireframe Modeling}

For rectilinear objects whose edges are straight lines and whose faces are planar, the ordered pair of vertices
 $\textbf{\mathfrak{V}} \in \mathbb{E}^3$ and edges $\textbf{\mathfrak{E}} \in \mathbb{E}^3$ denoted by $(\textbf{\mathfrak{V}} , \textbf{\mathfrak{E}})$ is the object's wireframe.  
In a practical sense, it is the skeleton of an object wherein joints are vertices
and bones are edges. In [6], a six ­step algorithm to generate an object's
wireframe was developed wherein an object's wireframe was represented by a vertex table and an edge table. 
Although the work in [6] had drawbacks such as not checking the validity of input data, wireframe modeling has always provided designers with a chance to experiment with the final result of their models through sketching and it is frequently used to preview complex models. However, the use of only edge information left wireframe models ambiguous
on rectilinear polyhedra talk less of topological ones. Figure 2.1 below shows the wireframe of a sphere in greyscale.

%--------------------------------------------------------------------------------------------

\begin{figure}[htbp]
\centering
\includegraphics[trim=0.1cm 0.3cm 0.5cm 0.5cm, clip=true, totalheight=0.5\textheight]{Figures/Sphere.png}
\caption[A wireframe of a sphere]{A wireframe of a sphere}
\label{Sphere}
\end{figure}

%--------------------------------------------------------------------------------------------

%----------------------------------------------------------------------------------------
%	SURFACE MODELING
%----------------------------------------------------------------------------------------

\section{Surface Modeling}

After breakthroughs in wireframe modeling, research efforts in geometric modeling were directed 
towards extending the geometric coverage of CAD packages by incorporating complex free­form surfaces
 and curves. In this section, we emphasize on algebraic surfaces and curves used within BRL­-CAD 
as it is the basis for Bezier surfaces and NURBS.

%--------------------------------------------------------------------------------------------

\begin{figure}[htbp]
\centering
\includegraphics[trim=0.1cm 0.3cm 0.5cm 0.5cm, clip=true, totalheight=0.5\textheight]{Figures/Primitives.png}
\caption[BRL­CAD Solid Primitive Shapes]{BRL­CAD Solid Primitive Shapes}
\label{Primitives}
\end{figure}

%--------------------------------------------------------------------------------------------

Figure 2.2 above shows a collection of some primitives used within the BRL-­CAD package before the heart-­shaped primitive   was developed – several of which are implicitly and/or parameterically represented by algebraic equations.On the BRL-­CAD's   ideas page[7], there is a list of primitives which have not yet been implemented such as the Steiner surface, the ring 
cyclide surface, the quartoid, Wallis' conical edge solid, etc.  

%----------------------------------------------------------------------------------------------------
%	Implicit Representations
%----------------------------------------------------------------------------------------------------

\subsection{Implicit Representation}

An algebraic surface in $\mathbb{E}^3$ is expressed as the set of points satisfying an irreducible polynomial equation
\begin{equation*}
\textbf{g(x,y,z) = 0}
\end{equation*} in the unknowns x,y and z.\\
A polynomial \textit{\textbf{f(x,y,z)}} over a field $\textbf{\mathbb{F}}$ is said to be irreducible over $\textbf{\mathbb{F}}$ if the degree of \textit{\textbf{f(x,y,z)}} is positive and its only factors are \textit{c} and \textit{cf(x,y,z)} where \textit{c} is a non­zero constant in $\textbf{\mathbb{F}}$.

The requirement of irreducibility is so that a surface represented by an equation
should not be decomposed into two separate surfaces, each of which can be
described by an implicit equation.
 
% Chapter Template

\chapter{ANALYSIS AND DESIGN} % Main chapter title

\label{Analysis And Design} % Change X to a consecutive number; for referencing this chapter elsewhere, use \ref{ChapterX}

\lhead{Chapter 3. \emph{Analysis And Design}} % Change X to a consecutive number; this is for the header on each page - perhaps a shortened title

%----------------------------------------------------------------------------------------
%	SECTION 1
%----------------------------------------------------------------------------------------

\hspace{30} In this chapter,   we   state   our   aim   of   contributing   to   the   BRL-­CAD   project  
and   explain   how   we   implemented   a   heart­shaped   primitive   in   the   project   design  
section.   Firstly,   we   introduce   the   concept   of   free   and   open   source   software. Secondly,   we   do   an   overview   of   the   BRL-­CAD   software   package.   After,   we  state   our   aim   of   contributing   to   the   BRL-­CAD   project.   Finally,   we   give   a   detailed  explanation   of   the   design   which   we   employed   to   implement   the   heart-­shaped primitive for BRL-­CAD.

%------------------------------------------------------------------------------------------------------------------------------
%		OPEN SOURCE COMMUNITY
%------------------------------------------------------------------------------------------------------------------------------
\section{The Open Source Community}

\hspace{30} Depending   on   how   we   choose   to   call   it,   \textit{Free/Libre/Open   Source  
Software   (FLOSS)},   \textit{Free   and   Open   Source   Software   (FOSS)}   or   simply   \textit{Open  
Source   Software   OSS)}   is   software   for   which   users   have   access   to   both   the  
source   code   and   binary   executables   and   is   licensed   under   a   license   which  
permits   its   users   to   read,   edit   and   distribute   the   software   to   anyone   and   for   any  
reason.   This   distinguishes   open   source   software   from   commercial   software  
which   is   distributed   by   giving   away   its   binary   executable   version   only.   Usually,  
OSS   is   distributed   at   no   cost   with   limited   restrictions   on   how   it   can   be   used.  
According   to   Eric   S.   Raymond[34],   one   of   the   most   prominent   evangelists   of  
the   open   source   movement,   hackerdom   can   be   likened   to   what   anthropologists  
call   a   gift   culture–   a   culture   wherein   members   gain   status   and   reputation   by  
giving   away   their   time,   creativity   and   skills   to   reading,   writing   and   debugging  
software,   publishing   useful   information   in   blogs   or   documents   like   Frequently  
Asked   Questions   (FAQs)   lists   as   well   as   handling   unglamorous   tasks   like  
maintaining   mailing   lists,   moderating   news   groups,   etc   without   any   monetary  
compensation.   The   word   \textit{hacker}   was   coined   by   a   shared   community   of   expert  
programmers   and   networking   masters   which   traces   its   history   back   to   the   days  
of   the   earliest   ARPAnet   experiments   and   time­sharing   minicomputers   who  
made   the   unix   operating   systems   and   the   world­wide   web   work.   As   opposed   to  
hackers,   \textit{crackers}   who   are   more   interested   in   breaking   software   and   perturbing  
phone systems.   

\hspace{30} Today,   the   open   source   community   has   become   a   self­organizing  
collaborative   social   network   of   hackers   driven   by   a   passion   to   solve   problems  
using   free   software   with   thousands   of   projects   hosted   on   Sourceforge[35]   and  
Github[36].   It   has   singularly   developed   some   software   packages   and   tools  
which   are   the   best   in   the   world   such   as   the   firefox   web   browser,   Apache  
web­server,   Linux   operating   systems   like   BSD,   Ubuntu,   Debian,etc,   the   MySQL  
database   management   system,   the   VLC   media   player,   programming  languages   and   tools   like   gcc,   C,   Perl,   Python,   Java,   etc   and   much   more.   Some  Examples   of   CAD   packages   within   the   open   source  community   include   BRL­CAD,   Blender,   FreeCAD,   openSCAD   and   LibreCAD,  
etc.

%--------------------------------------------------------------------------------------------------------------------------------

%--------------------------------------------------------------------------------------------------------------------------------
%	ANALYSIS OF THE WORK
%--------------------------------------------------------------------------------------------------------------------------------
\section{Analysis Of The Work}

\hspace{30} BRL­-CAD   (   pronounced   Be­-Are-­El-­CAD)   was   originally   conceived   and  
written   by   the   late   Mike   Muss,   a   programmer   and   networking   expert   who   also  
wrote   the   popular   PING   network   program.   In   1979,   the   United   States   Army's  
Ballistic   Research   Laboratory   (BRL)   (the   agency   responsible   for   creating  
ENIAC,   the   world's   first   general­purpose   electronic   computer   in   the   1940s)  
identified   a   need   for   tools   that   could   assist   with   the   computer   simulations   and  
analysis   of   combat   vehicle   systems   and   environments.   When   no   existing   CAD  
package   was   found   to   be   adequate   for   this   specialized   purpose,   Mike   and  
fellow   software   developers   began   developing   and   assembling   a   unique   suite   of  
utilities   capable   of   interactively   displaying,   editing,   and   interrogating   geometric  
models.   Those   early   efforts   subsequently   became   the   foundation   on   which  
BRL­CAD was built.  

\hspace{30} The   initial   architecture   and   design   of   BRL-­CAD   began   in   1979   and   its  
development   as   a   unified   software   package   kicked   off   in   1983   with   its   first  
public   release   the   following   year.   As   a   software   package   with   a   mature   code  
base   which   has   been   actively   developed   for   decades,   BRL-­CAD   pays   close  
attention   to   design   and   maintainability.   Like   other   FLOSS   packages,  
BRL­-CAD's   source   code   and   most   of   its   project   data   are   stored   in   a  
subversion   version   control   system   for   change   tracking,   collaborative  
development   and   is   redistributed   as   free   and   open   source   software   under   the  
Open   Source   Initiative   license   terms.   The   design   of   its   system   architecture   is  
based   on   a   unix­methodology   of   command   of   the   command­line   services,  
providing   many   tools   that   work   in   harmony   to   complete   a   specific   task.   These  
tools   include   geometry   and   image   converters,signal   and   image   processing  
tools,   various   ray   tracing   applications,geometry   manipulators,   and   much   more.  
They   will   also   be   used   to   test   that   the   geometric   properties   of   the   heart­shaped  
primitive works as we will see in Chapter Four.

\hspace{30} The   basic   layout   of   its   code   places   public   API   headers   in   the   top­level  
\textit{\textbf{include/}}   directory   and   source   code   for   both   applications   and   libraries   in   the  
\textit{\textbf{src/}}   directory.   The   following   is   a   partial   listing   of   how   BRL-­CAD's   source   code  
is organised in a typical checkout or source distribution. 

\textbf{Applications and Resources}  

\begin{itemize} 
\item✦ \textit{\textbf{db/}} for Example Geometry.  
\item✦ \textit{\textbf{doc/}} for project Documentation.  
\item✦ \textit{\textbf{include/}} for Public API headers.  
\item✦ \textit{\textbf{regress/}} for Regression test scripts  
\item✦ \textit{\textbf{src/}} for Application and library source.  
\item✦ \textit{\textbf{src/conv}} for Geometry converters.  
\item✦ \textit{\textbf{src/fb}} for Displaying data in windows.  
\item✦ \textit{\textbf{src/mged}} for the Multi­device geometry editor, the main GUI application.   
\item✦ \textit{\textbf{src/rt}} for Ray tracing applications.  
\item✦ \textit{\textbf{src/util}} for Image processing utilities.
\end{itemize}  

\textbf{Libraries}
  
\begin{itemize} 
\item✦ \textit{\textbf{src/libbn}} for Numerics library.  
\item✦ \textit{\textbf{src/libbu}} for Utility library.  
\item✦ \textit{\textbf{src/libgcv}} for Geometry conversion library.  
\item✦ \textit{\textbf{src/libged}} for Geometry Editing library.  
\item✦ \textit{\textbf{src/icv}} for Image conversion library.  
\item✦ \textit{\textbf{src/libpkg}} for Network Package library.  
\item✦ \textit{\textbf{src/librt}} for Ray­tracing library.  
\item✦ \textit{\textbf{src/libwbd}} for Geometry creation library.
\end{itemize}

\hspace{30} The   majority   of   BRL-­CAD's   source   code   is   written   in   ANSI/POSIX   C   with   the  
intent   of   strictly   conforming   with   the   C   standard.   The   core   libraries   are   all   C   API  
though   several   such   as   the   Utility   and   Ray­tracing   libraries   use   C++   for  
implementation   details.   Major   components   of   the   system   are   written   in   C,   C++,  
Tcl/Tk,   Bash   and   Php   with   source   code   files   using   extensions   such   as   *.c,   *.h,  
*.cpp,   *.tcl,   *.tk,   *.sh   and   *.php.   BRL­CAD   uses   the   CMake   build   system   for  
compilation and an inbuilt testing infrastructure in regress/ for unit testing.  

\hspace{30} BRL-­CAD   has   a   long­lasting   heritage   of   maintaining   verifiable,   validated  
and   repeatable   results   in   critical   portions   of   the   software   package,   particularly  
within   the   ray   tracing   library.   It   has   an   inbuilt   testing   infrastructure   which  
compares   all   program   output   against   benchmark   results   during   each   build.   The  
ray   tracing   library   is   a   multi­representational   library   which   lies   at   the   heart   of  
BRL­-CAD   and   uses   a   suite   of   other   libraries   for   other   basic   application  
functionality.   Considerable   attention   is   put   into   verification   and   validation  
throughout   the   package   which   includes   regression   tests   that   compare   runtime  
behaviour   against   known   results   and   reports   any   adverse   variances   from  
standard results as failures.  

\hspace{30} Despite   this   sophisticated   infrastructure,   performant   design   and  
long­lasting   heritage,   many   still   coin   BRL-­CAD's   aspiration   of   one   day   being   the  
most   widely   used   open   source   CAD   package   as   rather   lofty  
for the following reasons;

\begin{itemize}
\item With   one   of   the   fastest   ray­tracers   in   existence   (on   several   types   of  
geometry)   which   is   supported   by   an   effective   Laguerre-­based   root   solver  
and   used   within   academia   for   scientific   instruction,   computer   graphics  
education   and   research,   the   stability   of   BRL­-CAD's   root   solver   on  
higher ­order   polynomials   such   as   quintics   (of   power   5)   and   sextics   (of  
power 6) is still uncertain.

\item As   an   open   source   CAD   software   which   is   deeply   rooted   in   the  
Constructive   Solid   Geometry,   BRL-­CAD's   set   of   primitives   is   still   limited  
to traditional ones such as cones, cylinders, spheres, tori, etc.

\item BRL­-CAD   is   widely   used   within   agencies   within   the   United   States  
Government   for   the   modeling   of   military   artillery   and   simulating   combat  vehicle   systems   and   environments.   This   severely   limits   is   user   base   to  the   military   sector   and   gives   it   an   “­warring”   flare   which   repels   users   who  would have used it for more entertainment purposes.
\end{itemize}

In   a   bid   to   solve   the   aforementioned   problems,   we   embarked   on   a   journey   to  
develop   a   heart-­shaped   primitive   for   BRL­-CAD.   In   the   following   section,   we  
document   how   we   wrote   various   callback   functions   which   compute   useful  
geometric   properties   for   the   heart­shaped   primitive   such   as   formatted  
description,   database   importation   and   exportation,   computation   of   the   bounding  
box, plotting the wireframe and ray tracing.

%---------------------------------------------------------------------------------------------------------------------------------
%	DESIGN
%---------------------------------------------------------------------------------------------------------------------------------

\section{Design}

\hspace{30} After   having   stated   our   goal   of   contributing   to   the   BRL-­CAD   project,   we  
now   explain   how   we   implemented   the   heart­shaped   primitive.   Currently,  
BRL­-CAD   aspires   to   become   the   most   widely   used   open   source  
CAD   software   package   in   existence.   Presently,   it   is   mostly  
used   by   the   United   States   of   America's   government   agencies   which   fund   its  
development   as   well   as   academic   institutes   which   use   use   it   for   computer  
graphics   educations   and   scientific   research.   Aljazeera   news   Channel's   recent  
revelation   that   less   than   1\%   of   the   world's   population   works   in   the   military   sector  
is   indicative   of   the   fact   that   BRL-­CAD's   usage   must   go   beyond   the   military  
sector to break the status­ quo.   

%-----------------------------------------------------------------------------------------------------------------------------------
%----------------------------------------------------------------------------------------
%	SECTION 2
%----------------------------------------------------------------------------------------

\section{Main Section 2}

Sed ullamcorper quam eu nisl interdum at interdum enim egestas. Aliquam placerat justo sed lectus lobortis ut porta nisl porttitor. Vestibulum mi dolor, lacinia molestie gravida at, tempus vitae ligula. Donec eget quam sapien, in viverra eros. Donec pellentesque justo a massa fringilla non vestibulum metus vestibulum. Vestibulum in orci quis felis tempor lacinia. Vivamus ornare ultrices facilisis. Ut hendrerit volutpat vulputate. Morbi condimentum venenatis augue, id porta ipsum vulputate in. Curabitur luctus tempus justo. Vestibulum risus lectus, adipiscing nec condimentum quis, condimentum nec nisl. Aliquam dictum sagittis velit sed iaculis. Morbi tristique augue sit amet nulla pulvinar id facilisis ligula mollis. Nam elit libero, tincidunt ut aliquam at, molestie in quam. Aenean rhoncus vehicula hendrerit.

% Chapter Template

\chapter{RESULTS \& DISCUSSION} % Main chapter title

\label{Results \& Discussion} % Change X to a consecutive number; for referencing this chapter elsewhere, use \ref{ChapterX}

\lhead{Chapter 4. \emph{Results \& Discussion}} % Change X to a consecutive number; this is for the header on each page - perhaps a shortened title

%----------------------------------------------------------------------------------------
%	SECTION 1
%----------------------------------------------------------------------------------------

\hspace{30} In   this   chapter,   we   present   and   discuss   the   results   obtained   from   our  
work.   We   explain   how   we   tested   the   different   geometric   properties   of   the  
heart­-shaped   primitive.   Without   loss   of   generality,   we   used   a   heart-­shaped  
object   centered   at   the   origin   (0,0,0), possesses   3   radial   vectors (5,0,0), (0,5,0) 
and (0,0,5) as   well   as   a   distance   to   cusps   of   4.   Let's   suppose   this  
object   called   \emph{amour}   and   is   stored   in   the   $heart\_example.g$   database. 
Given the situation of our work in the field of CAD, we   use   images   to   demonstrate   that 
  each   of   the   heart­-shape's   geometric  properties work.  

%----------------------------------------------------------------------------------------

\section{Type-in support for the heart­-shaped primitive}

\hspace{30} During   the   modeling   process,   a   designer   using   BRL­-CAD   creates objects   by   typing   
its   parameters   into   either   the   mged   or   archer   graphical   user  interfaces.   
Having   built   this   capacity   into   the   heart-­shaped   primitive,   we   used  
BRL-­CAD's \textbf{\textit{in}}[38] command to test its mettle.  

The   \textit{\textbf{in}}   command   enables   the   user   to   type   in   the   arguments   needed   to   create   a  shape   alongside   its   name   and   type.   It   supports   various   options   and   may   be  
invoked   with   no   arguments.   The   \textit{\textbf{-­s}}   option   invokes   the   primitive   edit   mode   on   a  
new   object   immediately   it   is   created.   In   order   to   test   the   type   in   support   of   the  
heart­-shaped   primitive,   we   create   the   \textit{amour}   object   by   typing   its   name,   type   and  
other   parameters   into   the   archer   interface.   After   opening   the   $heart\_example.g$  
database   database   destined   to   hold   the   \textit{amour}   object,   we   type   \textit{\textbf{“in   amour   hrt   0  
0   0   5   0   0   0   5   0   0   0   5   4”}}   into   archer's   command   line   interface.   The   object's  
name,   \textit{amour},   is   printed   on   the   command ­line   indicating   that   the   object   has  
indeed   been   created. Figure   4.1 below   shows   that   the   \textit{amour}  
heart-­shape object can be created by typing in its parameters through the keyboard.

\begin{figure}[htbp]
\centering
\includegraphics[trim=0.0cm 0.5cm 0.1cm 0.1cm, clip=true, totalheight=0.4\textheight]{Pictures/Typein.png}
\caption[Testing type in support for the heart-­shaped primitive in archer]{Testing type in support for the heart­-shaped primitive in archer}
\label{Typein}
\end{figure}

%----------------------------------------------------------------------------------------------------------------
%	SUBSECTION 1
%----------------------------------------------------------------------------------------------------------------
\section{Formatted description of the heart­-shaped primitive}

After   having   created   objects   for   modeling,   it   sometimes   becomes  
necessary   to   display   a   description   of   these   objects.   For   us   to   test   that   we   can  
describe   the   heart­shaped   primitive,   we   used   BRL-­CAD's   \textit{\textbf{l}}[39]   command   on  
the   amour   object.   The   \textit{\textbf{l (listing)}}   command   displays   a   verbose   description   of   a
 specific   list   of   objects.   If   the   shape   of   the   object   is   a   primitive,   then   detailed  
parameters   of   that   shape   are   displayed.   If   the   object   is   a   combination   of   other  
primitives,   then   the   boolean   formula   for   the   combination   is   listed   while   indicating  
any   accumulated   transformations.   If   a   shader   and   colour   has   been   assigned   to  
the   combination,   then   all   details   will   be   listed.   The   \textit{\textbf{-­t}}   (terse)   option   displays   a  
shorter list of primitive shape parameters.

\hspace{30} To   describe   the   \textit{amour}   object,   we   print   amour's   parameters   in   both   terse  
and   verbose   forms   by   running   the   \textit{\textbf{“l -t amour”}}   and   \textit{\textbf{“l amour”}}   commands  
respectively in the archer command prompt. This is shown in Figure 4.2 below.

\begin{figure}[htbp]
\centering
\includegraphics[trim=0.0cm 0.5cm 0.1cm 0.1cm, clip=true, totalheight=0.4\textheight]{Pictures/Describe.png}
\caption[Testing the formatted description of the heart­-shaped primitive]{Testing the formatted description of the heart­-shaped primitive}
\label{Describe}
\end{figure}

%-----------------------------------
%	SUBSECTION 2
%-----------------------------------

\section{The bounding box of the heart-­shaped primitive}

As   we   earlier   stated,   the   $rt\_hrt\_bbox()$   function   was   implemented   to
 calculate   the   bounding   box   of   the   heart-­shaped   primitive.   In   order   to   test   that  
the   bounding   box   of   the   heart-­shaped   primitive   is   computed,   we   use  
BRL­-CAD's \textit{\textbf{bb}}[40] command.

\hspace{30} The   \textit{\textbf{bb (bounding box)}}  command   reports   dimensional   information   about   objects   using  
bounding   boxes.   It   does   this   by   calculating   an   axis­aligned   bounding   box   for   an  
object   and   printing   the   dimensions   of   that   box   to   the   command   prompt   of  
archer.   The   \textit{\textbf{bb}}   command   support   various   options,   most   of   which   control   the  
type of information reported.  

\begin{itemize}
\item The \textit{\textbf{-­e} (extent)} option reports the extent of the bounding box by printing its minimal and maximal points.  
\item The \textit{\textbf{-­d} (default)} option reports the length, width and height of the box.  
\item The \textit{\textbf{-­v} (volume)} option prints the volume of the bounding box by default too.  
\item The \textit{\textbf{-­q} (quiet)} option prints the properties of an object in quiet mode by disabling the printing of the default header.
\end{itemize}

\hspace{30} Once   more   we   use   the   amour   object   and   the   \textit{bb}   command   to   test   how  
effectively   the   bounding   box   of   the   heart­-shaped   primitive   was   implemented.  
To   report   the   extent   of   the   bounding   box   of   the   amour   object,   we   print   its  
minimal   and   maximal   points   by   running   the   \textit{\textbf{“bb   ­-qe   amour”}}   command   in   either  
the   mged   or   archer   command   prompts.   Then,   we   ran   the   \textit{\textbf{“bb   -­qv   amour”}}  
command   in   archer   so   that   the   volume   of   the   amour   object   is   reported   in   cubic  
millimeters.   After,   we   reported   the   length,   width   and   height   of   amour's   bounding  
box   by   running   the   \textit{\textbf{“bb   ­-qd   amour”}}.   Finally,   to   report   the   volume   and  
dimensions   of   the   bounding   box,   we   ran   the   \textit{\textbf{“bb   amour”}}   command   in   archer's  
command prompt. 

\hspace{30} The image above in Figure 4.3 shows the results obtained after running the aforementioned commands. 

\begin{figure}[htbp]
\centering
\includegraphics[trim=0.0cm 0.5cm 0.1cm 0.1cm, clip=true, totalheight=0.4\textheight]{Pictures/Bounding.png}
\caption[Testing the bounding box of the heart­-shape]{Testing the bounding box of the heart­-shape}
\label{Bounding}
\end{figure}

\section{Plotting the wireframe of the heart­-shaped primitive}

\hspace{30} The   process   of   modeling   sometimes   warrants   the   preview   of   the   skeleton  
of   a   specific   set   of   objects.   To   test   that   the   wireframe   of   the   heart­-shaped  
primitive   is   working,   we   use   the   \textit{\textbf{draw}}[41]   command   in   BRL­-CAD.   The   \textit{\textbf{draw}}  
command   displays   objects   in   either   the   mged   or   archer   interfaces.   It   is  
synonymous   to   BRL­-CAD's   \textit{\textbf{e}}   command.   The   draw   command's   \textit{\textbf{-­C} (colour)}  
option   enables   the   user   to   specify   a   colour   that   overrides   all   other   previous  
colour specifications.  

\hspace{30} To   draw   the   wireframe   of   the   amour   object   using   white   wires,   we   run   the  
\textit{\textbf{“draw -­C 255/255/255 amour”}}   command.   The   image   in   Figure   4.4   below   shows  
the wireframe of amour with red iso­contours.  

\begin{figure}[htbp]
\centering
\includegraphics[trim=0.0cm 0.5cm 0.1cm 0.1cm, clip=true, totalheight=0.4\textheight]{Pictures/Wireframe.png}
\caption[Testing the wireframe of the heart­shaped primitive]{Testing the wireframe of the heart­shaped primitive}
\label{Wireframe}
\end{figure}

%----------------------------------------------------------------------------------------
%	SECTION 2
%----------------------------------------------------------------------------------------

\section{Main Section 2}

Sed ullamcorper quam eu nisl interdum at interdum enim egestas. Aliquam placerat justo sed lectus lobortis ut porta nisl porttitor. Vestibulum mi dolor, lacinia molestie gravida at, tempus vitae ligula. Donec eget quam sapien, in viverra eros. Donec pellentesque justo a massa fringilla non vestibulum metus vestibulum. Vestibulum in orci quis felis tempor lacinia. Vivamus ornare ultrices facilisis. Ut hendrerit volutpat vulputate. Morbi condimentum venenatis augue, id porta ipsum vulputate in. Curabitur luctus tempus justo. Vestibulum risus lectus, adipiscing nec condimentum quis, condimentum nec nisl. Aliquam dictum sagittis velit sed iaculis. Morbi tristique augue sit amet nulla pulvinar id facilisis ligula mollis. Nam elit libero, tincidunt ut aliquam at, molestie in quam. Aenean rhoncus vehicula hendrerit.
 
%-----------------------------------------------------------------------------------------------------------------
\chapter{CONCLUSIONS}
\label{Conclusions}
\lhead{Chapter 5. \emph{Conclusions}} 
%-----------------------------------------------------------------------------------------------------------------

\hspace{30} In   this   chapter,   we   note   the   new   and   original   contributions   of   our   work   on  
the   heart-­shaped   primitive   to   the   field   of   CAD   and   highlight  
possible   research   orientations   which   can   spring   up   from   it.   During   our   work,   we  
first   designed   the   heart­-shaped   primitive's   structure,   wrote   necessary   callback  
functions and tested them using BRL-­CAD's inbuilt testing infrastructure. 

%----------------------------------------------------------------------------------------
%	SECTION 1
%----------------------------------------------------------------------------------------

\section{Our Contribution}

The most important contributions of this thesis are outlined below;  

\begin{itemize}
\item We   showed   that   BRL-­CAD's   Laguerre-­based   root   solver   is   indeed   a  
sure­fire   iterative   method   for   finding   roots   of   polynomials   and   ascertained  
it's stability on sextic equations.  
\item This   work   provides   a   guideline   for   the   development   of   primitives   within  
open   source   CAD   software   by   highlighting   the  implementation   of   
geometrically­ useful properties for the heart-shaped primitive within BRL-­CAD.  
\item By   producing   animated   videos   from   resulting   rendered   images,   we  
showed   that   BRL­-CAD   is   a   software   package   which   artists   can   use   to  
produce   cartoon   animations,   design   cards,royal   seals   and   banners,   gifts  
and   presents   for   family   and   communal   celebrations   such   as   weddings,  
valentine's day, etc, for entertainment purposes.
\end{itemize}

%--------------------------------------------------------------------------------------------------
%		SECTION 2
%--------------------------------------------------------------------------------------------------

\section{Further Research}

\hspace{30} In   order   to   invite   more   artists   to   use   BRL-­CAD   for   the   creation   of  
entertainment   content   and   enhance   BRL-­CAD's   functionality   so   that   it   becomes  a 
better   open   source   CAD   package,   we   propose  the following research ideas;

\begin{itemize}
\item Although   the   heart-­shaped   primitive   has   been   shown   to   have   possess   a  
wireframe   and   an   implicit   surface   representation,   it   would   be   nice   for   this  
primitive   to   own   an   explicit   surface   representation   based   on   its   parametric  
equations.  
\item BRL­-CAD   is   used   within   the   scientific   community   for   research   and  
computer   graphics   education,   it   would   be   nice   to   investigate   the   stability  
of   BRL-­CAD's   Laguerre­-based   root   solver   on   septic,   octic   and   nonic  
equations which are monic polynomials of order 7, 8 ans 9 respectively.  
\item Since   BRL-­CAD   boasts   of   more   than   400   commands   within   its   pocket,   it  
would   be   worthwhile   to   invite   more   artists   within   the   entertainment   industry  
by   adding   more   functionality   to   its   animate   command   so   that   several  
images can be combined into animated videos.
\end{itemize}

\clearpage

%--------------------------------------------------------------------------------------------------------
 

\clearpage % Start a new page

%----------------------------------------------------------------------------------------
%	BIBLIOGRAPHY/REFERENCES
%----------------------------------------------------------------------------------------
\newgeometry{a4paper,lmargin=2.6in,rmargin=0in,tmargin=1.3in,bmargin=0in}
\label{References}

\lhead{\emph{References}} % Name the Page header "References"

\bibliographystyle{ieeetr} % Use the "ieeetr" BibTeX style for formatting the References

\bibliography{References} % The references information are stored in the file named "References.bib"

%----------------------------------------------------------------------------------------
%	THESIS CONTENT - APPENDIX
%----------------------------------------------------------------------------------------

\backmatter

\appendix % Tell LaTeX that the following 'chapters' are Appendices

\lhead{\emph{Appendix}}

\chapter{APPENDIX} % Main appendix title

\label{Appendix} % For referencing this appendix elsewhere, use \ref{AppendixA}

\lhead{\emph{Appendix}} % This is for the header on each page - perhaps a shortened title

Coefficients of the terms in equation \textit{(4)} \\

$C_6  = 4320a^2b^2c^2 + 960a^2c^2(a^2 + c^2) + 320(a^6 + c^6) + 2160b^2(a^4 + c^4) + 4860b^4(a^2 
+ c^2) - 36b^3c^3 + 3645b^6$
  
$C_5  = 8640(ax_0b^2(a^2 + c^2) + by_0a^2c^2  + cz_0b^2(a^2 + c^2)) + 9720b^4(ax_0 + cz_0) + 1920(a^4 
+ c^4(ax_0 + cz_0)) + 4320(a^4 + c^4by_0) + 3840(a^2 + c^2)(ax_0 + cz_0) + 19440(c^2b^2 + ab^2) 
by_0 -­ 108bc(c^2by_0 + cz_0b^2) -­ 320a^2ba^3 + 21870y_0b^5$  

$C_4  = 4320(a^2(b^2z_0^2 + c^2y_0^2) + b^2(x_0^2c^2 - a^2 - c^2)) -­ 960(a^4(1 -­ z_0^2) + c^4 + b^2(a^2y_0 ­- 
b^2x_0^2)) + 4860(b^4(x_0^2 + z_0^2 -­ 1)) + 4800(a^4x_0^2 + c^4z_0^2)+ 2160y_0^2(a^4 + c^4) + 
5760a^2c^2(x_0^2 + z_0^2) + 7680ax_0cz_0(a^2 + c^2) + 17280((ax_0by_0 + by_0cz_0 )(a^2 + c^2) + 
ax_0cz_0b^2) + 12960b^2(a^4x_0^2 + c^4z_0^2) + 29160b^2y_0^2(a^2 + c^2) -­ 108bc(b^2z_0^2 + c^2y_0^2)
- 1920a^2c^2 -­ 54675y_0^2b^4 -­ 324z_0c^2b^2y_0 - 640ax_0b^3$    

$C_3 = 3840(ax_0 + cz_0)(a^2z_0^2 + c^2x_0^2 - a^2 + c^2) + 8640(ax_0(z_0^2b^2 + c^2y_0^2 + b^2x_0^2 + 
a^2y_0^2 -­ b^2) + by_0(a^2z_0^2 + c^2x_0^2 - a^2 - c^2) + cz_0(a^2y_0^2 + b^2x_0^2 + bz_0^2 + c^2y_0^2 - ­ b^2)) + 6400 
(a^3x_0^3 + c^3z_0^3) + 19440by_0(b^2(x_0^2 + z_0^2 -­ 1) + y_0^2(a^2 + c^2)) -­ 36(b^3z_0^3 + c^3y_0^3 ) + 
11520(ax_0cz_0(ax_0 + cz_0))­ - 324by_0cz_0(bz_0 + c + y_0) + 25920by_0(z_0^2c^2 + x_0^2a^2) ­- 
320x_0^2b^3  + 72900y_0^3b^3 + 34560ax_0by_0cz_0 + 58320ax_0b^2y_0^2 -­ 1920ax_0b^2y_0 + 
58320cz_0b^2y_0  -­ 960a^2by_0^2$

$C_2 = 960(a^2(1 + z_0^4) +  c^2(1 + x_0^4) - b^2x_0^2y_0) + 2160b^2(1 + x_0^4 + z_0^4) + 4320(y_0^2( 
a^2 + c^2 + a^2z_0^2 + c^2x_0^2) + b^2(x_0^2z_0^2 - x_0^2­ - z_0^2)) + 4800(x_0^3a^2 + z_0^3c^2)
 + 4860y_0^4(a^2 + c^2) + 5760(z_0^2c^2(1 + x_0^2) + x_0^2a^2(z_0^2 - ­1)) + 7680ax_0cz_0(x_0^2 + z_0^2 - 1­ - 1920(a^2z_0^2 + c^2x_0^2 + ax_0y_0^2b) + 17280(by_0(ax_0^3 - ax_0 + cz_0^3 -­ cz_0 + az_0^2) + ax_0cz_0y_0^2) + 38880bz_0^3(ax_0 + cz_0) + 12960y_0^2(a^2x_0^2 ­- c^2z_0^2) + 
29160b^2y_0^2(x_0^2 + z_0^2 -­ 1) ­- 180y_0z_0c^2y_0^2 + b^2z_0^2) ­- 320a^2y_0^3 + 54675y_0^3b^2$  

$C_1  = 1920((ax_0 + cz_0)(1 + x_0^4 + z_0^4)) + 4320by_0(1 + x_0^4 + z_0^4) + 8640(-­by_0z_0^2 + 
ax_0z_0^2y_0^2 + cz_0x_0^2y_0^2 + by_0^2x_0z_0^2 -­ ax_0y_0^2 + ay_0^2 + ax_0^3y_0^2 -­ cz_0y_0^2 + cz_0^3y_0^2 - by_0x_0^2) -­ 
3840(cz_0^3 + ax_0^3 + ax_0z_0^2 - ­ax_0^2z_0^2 + cz_0x_0^2 -­ cz_0^3x_0^2) -­ 19440(by_0^3(1 -­ x_0^2 -­ z_0^2)) + 
9720(y_0^4(ax_0 + cz_0)) + 21870(by_0^4)­ - 640(ax_0y_0^3 - by_0^2x_0^2) -­ 108(cz_0^2y_0^2 + by_0^2y_0^3)$
 
$C_0 = 320(-­x_0^6 + z_0^6 - x_0^2y_0^3) + 960(x_0^2 + z_0^4x_0^4 + x_0^2z_0^4 -­ x_0^4 -­ z_0^4) + 4320(x_0^2y_0^2z_0^2 -­ 
x_0^2y_0^2 -­ z_0^2y_0^2) ­- 1920x_0^2z_0^2 - ­36x_0^3y_0^3 + 2160(y_0^2(1 + x_0^4 + z_0^4)) + 4860(y_0^4(x_0^2 + z_0^2  
+ 1)) + 3645y_0^5$


\end{document}  
