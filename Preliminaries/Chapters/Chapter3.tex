% Chapter Template

\chapter{ANALYSIS AND DESIGN} % Main chapter title

\label{Analysis And Design} % Change X to a consecutive number; for referencing this chapter elsewhere, use \ref{ChapterX}

\lhead{Chapter 3. \emph{Analysis And Design}} % Change X to a consecutive number; this is for the header on each page - perhaps a shortened title

%----------------------------------------------------------------------------------------
%	SECTION 1
%----------------------------------------------------------------------------------------

\hspace{30} In this chapter,   we   state   our   aim   of   contributing   to   the   BRL-­CAD   project  
and   explain   how   we   implemented   a   heart­shaped   primitive   in   the   project   design  
section.   Firstly,   we   introduce   the   concept   of   free   and   open   source   software. Secondly,   we   do   an   overview   of   the   BRL-­CAD   software   package.   After,   we  state   our   aim   of   contributing   to   the   BRL-­CAD   project.   Finally,   we   give   a   detailed  explanation   of   the   design   which   we   employed   to   implement   the   heart-­shaped primitive for BRL-­CAD.

%------------------------------------------------------------------------------------------------------------------------------
%		OPEN SOURCE COMMUNITY
%------------------------------------------------------------------------------------------------------------------------------
\section{The Open Source Community}

\hspace{30} Depending   on   how   we   choose   to   call   it,   \textit{Free/Libre/Open   Source  
Software   (FLOSS)},   \textit{Free   and   Open   Source   Software   (FOSS)}   or   simply   \textit{Open  
Source   Software   OSS)}   is   software   for   which   users   have   access   to   both   the  
source   code   and   binary   executables   and   is   licensed   under   a   license   which  
permits   its   users   to   read,   edit   and   distribute   the   software   to   anyone   and   for   any  
reason.   This   distinguishes   open   source   software   from   commercial   software  
which   is   distributed   by   giving   away   its   binary   executable   version   only.   Usually,  
OSS   is   distributed   at   no   cost   with   limited   restrictions   on   how   it   can   be   used.  
According   to   Eric   S.   Raymond[34],   one   of   the   most   prominent   evangelists   of  
the   open   source   movement,   hackerdom   can   be   likened   to   what   anthropologists  
call   a   gift   culture–   a   culture   wherein   members   gain   status   and   reputation   by  
giving   away   their   time,   creativity   and   skills   to   reading,   writing   and   debugging  
software,   publishing   useful   information   in   blogs   or   documents   like   Frequently  
Asked   Questions   (FAQs)   lists   as   well   as   handling   unglamorous   tasks   like  
maintaining   mailing   lists,   moderating   news   groups,   etc   without   any   monetary  
compensation.   The   word   \textit{hacker}   was   coined   by   a   shared   community   of   expert  
programmers   and   networking   masters   which   traces   its   history   back   to   the   days  
of   the   earliest   ARPAnet   experiments   and   time­sharing   minicomputers   who  
made   the   unix   operating   systems   and   the   world­wide   web   work.   As   opposed   to  
hackers,   \textit{crackers}   who   are   more   interested   in   breaking   software   and   perturbing  
phone systems.   

\hspace{30} Today,   the   open   source   community   has   become   a   self­organizing  
collaborative   social   network   of   hackers   driven   by   a   passion   to   solve   problems  
using   free   software   with   thousands   of   projects   hosted   on   Sourceforge[35]   and  
Github[36].   It   has   singularly   developed   some   software   packages   and   tools  
which   are   the   best   in   the   world   such   as   the   firefox   web   browser,   Apache  
web­server,   Linux   operating   systems   like   BSD,   Ubuntu,   Debian,etc,   the   MySQL  
database   management   system,   the   VLC   media   player,   programming  languages   and   tools   like   gcc,   C,   Perl,   Python,   Java,   etc   and   much   more.   Some  Examples   of   CAD   packages   within   the   open   source  community   include   BRL­CAD,   Blender,   FreeCAD,   openSCAD   and   LibreCAD,  
etc.

%--------------------------------------------------------------------------------------------------------------------------------

%-----------------------------------
%	SUBSECTION 1
%-----------------------------------
\subsection{Subsection 1}

Nunc posuere quam at lectus tristique eu ultrices augue venenatis. Vestibulum ante ipsum primis in faucibus orci luctus et ultrices posuere cubilia Curae; Aliquam erat volutpat. Vivamus sodales tortor eget quam adipiscing in vulputate ante ullamcorper. Sed eros ante, lacinia et sollicitudin et, aliquam sit amet augue. In hac habitasse platea dictumst.

%-----------------------------------
%	SUBSECTION 2
%-----------------------------------

\subsection{Subsection 2}
Morbi rutrum odio eget arcu adipiscing sodales. Aenean et purus a est pulvinar pellentesque. Cras in elit neque, quis varius elit. Phasellus fringilla, nibh eu tempus venenatis, dolor elit posuere quam, quis adipiscing urna leo nec orci. Sed nec nulla auctor odio aliquet consequat. Ut nec nulla in ante ullamcorper aliquam at sed dolor. Phasellus fermentum magna in augue gravida cursus. Cras sed pretium lorem. Pellentesque eget ornare odio. Proin accumsan, massa viverra cursus pharetra, ipsum nisi lobortis velit, a malesuada dolor lorem eu neque.

%----------------------------------------------------------------------------------------
%	SECTION 2
%----------------------------------------------------------------------------------------

\section{Main Section 2}

Sed ullamcorper quam eu nisl interdum at interdum enim egestas. Aliquam placerat justo sed lectus lobortis ut porta nisl porttitor. Vestibulum mi dolor, lacinia molestie gravida at, tempus vitae ligula. Donec eget quam sapien, in viverra eros. Donec pellentesque justo a massa fringilla non vestibulum metus vestibulum. Vestibulum in orci quis felis tempor lacinia. Vivamus ornare ultrices facilisis. Ut hendrerit volutpat vulputate. Morbi condimentum venenatis augue, id porta ipsum vulputate in. Curabitur luctus tempus justo. Vestibulum risus lectus, adipiscing nec condimentum quis, condimentum nec nisl. Aliquam dictum sagittis velit sed iaculis. Morbi tristique augue sit amet nulla pulvinar id facilisis ligula mollis. Nam elit libero, tincidunt ut aliquam at, molestie in quam. Aenean rhoncus vehicula hendrerit.
